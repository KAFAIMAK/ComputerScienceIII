\documentclass[12pt]{scrartcl}

%\usepackage[printwatermark,disablegeometry]{xwatermark}

%\usepackage{epsfig,amssymb}

\usepackage{xcolor}
\usepackage{graphicx}
\usepackage{epstopdf}
\usepackage{multirow}
\usepackage{colortbl} 


\definecolor{steelblue}{RGB}{70, 130, 180}
\definecolor{darkred}{rgb}{0.5,0,0}
\definecolor{darkgreen}{rgb}{0,0.5,0}
\usepackage{hyperref}
\hypersetup{
  letterpaper,
  colorlinks,
  linkcolor=red,
  citecolor=darkgreen,
  menucolor=darkred,
  urlcolor=blue,
  pdfpagemode=none,
  pdftitle={Syllabus},
  pdfauthor={Christopher M. Bourke},
  pdfkeywords={}
}

\usepackage{fullpage}
\usepackage{tikz}
\pagestyle{empty} %
\usepackage{subfigure}

\definecolor{MyDarkBlue}{rgb}{0,0.08,0.45}
\definecolor{MyDarkRed}{rgb}{0.45,0.08,0}
\definecolor{MyDarkGreen}{rgb}{0.08,0.45,0.08}

\definecolor{mintedBackground}{rgb}{0.95,0.95,0.95}
\definecolor{mintedInlineBackground}{rgb}{.90,.90,1}

\usepackage[newfloat=true]{minted}

\setminted{mathescape,
           linenos,
           autogobble,
           frame=none,
           framesep=2mm,
           framerule=0.4pt,
           %label=foo,
           xleftmargin=2em,
           xrightmargin=0em,
           %startinline=true,  %PHP only, allow it to omit the PHP Tags *** with this option, variables using dollar sign in comments are treated as latex math
           numbersep=10pt, %gap between line numbers and start of line
           style=default} %syntax highlighting style, default is "default"

\setmintedinline{bgcolor={mintedBackground}}
%doesn't work with the above workaround:
\setminted{bgcolor={mintedBackground}}
\setminted[text]{bgcolor={mintedBackground},linenos=false,autogobble,xleftmargin=1em}
%\setminted[php]{bgcolor=mintedBackgroundPHP} %startinline=True}
\SetupFloatingEnvironment{listing}{name=Code Sample}
\SetupFloatingEnvironment{listing}{listname=List of Code Samples}

\setlength{\parindent}{0pt} %
\setlength{\parskip}{.25cm}
\newcommand{\comment}[1]{}

\usepackage{amsmath}
\usepackage{algorithm2e}
\SetKwInOut{Input}{input}
\SetKwInOut{Output}{output}
%NOTE: you can embed algorithms in solutions, but they cannot be floating objects; use [H] to make them non-floats

\usepackage{lastpage}

%\usepackage{titling}
\usepackage{fancyhdr}
\renewcommand*{\titlepagestyle}{fancy}
\pagestyle{fancy}
%\renewcommand*{\titlepagestyle}{fancy}
%\fancyhf{}
%\rhead{~}
%\lhead{~}
\renewcommand{\headrulewidth}{0.0pt}
\renewcommand{\footrulewidth}{0.4pt}

\lhead{~}
\chead{~}
\rhead{~}
\lfoot{\Title\ -- Syllabus}
\cfoot{~}
\rfoot{\thepage\ / \pageref*{LastPage}}

\makeatletter
\title{Computer Science III}\let\Title\@title
\subtitle{Data Structures \& Algorithms\\Syllabus\\
{\small
\vskip1cm
Department of Computer Science \& Engineering \\
University of Nebraska--Lincoln}
\vskip-1cm}
%\author{Dr.\ Chris Bourke}
\date{CSCE 310 -- Summer 2021}
\makeatother

\begin{document}

\maketitle

%\newwatermark[allpages=true,scale=5,textmark=Draft]{},

\hrule

\begin{quote}
``Computer Science is no more about computers than astronomy is about telescopes.''\hfill --Edsger Dijkstra
\end{quote}

\begin{quote}

``If you want to be a good programmer you just program every day for two years. 
If you want to be a world class programmer you can program every day for ten 
years, or you could program every day for two years and take an algorithms 
class.''
\hfill ---Charles E. Leiserson
\end{quote}


\section{Course Info}

\textbf{Prerequisites}: CSCE 156 (Computer Science II) and CSCE 235 (Discrete Math)

\textbf{Description}: A review of algorithm analysis, asymptotic notation, 
and solving recurrence relations. Advanced data structures and their 
associated algorithms, heaps, priority queues, hash tables, trees, binary 
search trees, and graphs. Algorithmic techniques, divide and conquer, 
transform and conquer space-time trade-offs, greedy algorithms, dynamic 
programming, randomization, and distributed algorithms. Introduction to 
computability and NP-completeness.

\textbf{Credit Hours}: 3

\textbf{Textbook}: The \emph{recommended} text book for this course is
\emph{Introduction to the Design and Analysis of Algorithms} (any edition)
by Anany Levitin.  However, no text book is required as there are plenty 
of free online Data Structures and Algorithms resources:
\begin{itemize}
  \item My lecture notes: \url{cse.unl.edu/~cbourke/ComputerScienceThree.pdf}
  \item Open DSA: \url{https://opendatastructures.org/}
  \item \emph{Algorithms} by Jeff Erickson \url{http://jeffe.cs.illinois.edu/teaching/algorithms/}
\end{itemize}

\textbf{Postrequisites}: If you are a Computer Science or Computer 
Engineering major you will need to receive a C or better in this course 
to continue in the major.  

\section{Course Overview}

Computer Science is not programming. Rather, Computer Science is the 
mathematical modeling and study of what computation is--what 
problems have a computational solution and how efficient that solution 
can be. Thus, a strong foundation in mathematics is essential to your 
success as a computer scientist.  At the heart of computer science 
are fundamental, discrete structures which we will study in this 
course.  Specifically, you will learn many of the mathematical 
definitions, techniques, and ways of thinking that will be useful 
in Computer Science.

\subsection{Topics}

\begin{itemize}
  \item A review of algorithms, algorithm analysis and asymptotics
  \item Brute Force algorithms, backtracking, generating combinatorial objects
  \item Divide \& conquer techniques, repeated squaring, Karatsuba multiplication, Strassen's matrix multiplication, etc.
  \item Algorithms for linear systems
  \item Greedy Algorithms: Huffman coding
  \item Balanced Trees: Heaps, AVL Trees, 2-3 Trees
  \item Hash-based data structures
  \item Graph algorithms: DFS, BFS, MSTs, path finding, shortest path
  \item Dynamic Programming
  \item Computation and computability
\end{itemize}

\section{Schedule}

See Canvas

\section{Course Delivery}

For summer sessions this course is delivered online only in a (more-or-less) 
asynchronously manner.  
\begin{itemize}
  \item Daily lectures will be live streamed via 
  YouTube (\url{https://www.youtube.com/c/ChrisBourkeUNL/live}), however 
  the time is yet to be determined. 
  \begin{itemize}
    \item Recordings of the lectures will be available 
    immediately following so you can watch/rewatch at your convenience
    \item During the live broadcast, Piazza will be used for 
    questions/answers
  \end{itemize}
  \item For assignments that allow collaboration, you may use any medium
  you choose.  
  \begin{itemize}
    \item You can establish your own Zoom rooms to talk back and forth 
    and share a screen
    \item You may use discord or slack instead
    \item You can (in fact should) be using git to share code (but 
    only use private repos)
    \item There are (free) online IDEs that allow you both to type in the same 
    editor at the same time: \url{https://repl.it}, \url{https://ide.cs50.io/};
    a more extensive list: \url{https://gist.github.com/rouzbeh84/4bafc9fe4fe02edf506d11997c4674b0}
  \end{itemize}
  \item Written solutions must be submitted through canvas as PDFs
  \item Live office hours will be held online via zoom 
  \item Exams will still be run, but asynchronously as ``take home'' exams 
  that will be released the day-of.  You will have a limited but 
  \emph{flexible} time period to complete them and submit it electronically.  
  No collaboration will be allowed on the exams.
\end{itemize}

\section{Accommodations for Students with Disabilities}

%updated from https://www.unl.edu/ssd/content/syllabus-statement-faculty
% 2020/07/01

The University strives to make all learning experiences as 
accessible as possible. If you anticipate or experience 
barriers based on your disability (including mental health, 
chronic or temporary medical conditions), please let me know 
immediately so that we can discuss options privately. To 
establish reasonable accommodations, I may request that you 
register with Services for Students with Disabilities (SSD). 
If you are eligible for services and register with their 
office, make arrangements with me as soon as possible to 
discuss your accommodations so they can be implemented in a 
timely manner. SSD contact information:  117 Louise Pound 
Hall.; 402-472-3787

\section{Grading}

Grading will be based on assignments (both written and programming
portions) as well as two exams.

\begin{table}[h]
\centering
{\small
\setlength{\tabcolsep}{0.5em} % for the horizontal padding
\renewcommand{\arraystretch}{1.2}% for the vertical padding

\begin{tabular}{lrrr}
\hline
\rowcolor{steelblue!50} Category & Number & Points Each & Total \\
\hline
\rowcolor{steelblue!5}  Assignments       &  4  & 200 & 800 \\
\rowcolor{steelblue!10} Midterm           & 100 & 1   & 100 \\
\rowcolor{steelblue!5}  Final             & 100 & 1   & 100 \\
\hline
Total  & & & 1,000 
\end{tabular}
}
\end{table}

\subsection{Scale}

Final letter grades will be awarded based on the following 
standard scale. This scale may be adjusted upwards if the 
instructor deems it necessary based on the final grades only.  
No scale will be made for individual assignments or exams.

\begin{table}[h]
\centering
\begin{tabular}{p{1cm}c}
Letter Grade & Percent \\
\hline\hline
A+ & $\geq 97$ \\
A  & $\geq 93$ \\
A- & $\geq 90$ \\
B+ & $\geq 87$ \\
B  & $\geq 83$ \\
B- & $\geq 80$ \\
C+ & $\geq 77$ \\
C  & $\geq 73$ \\
C- & $\geq 70$ \\
D+ & $\geq 67$ \\
D  & $\geq 63$ \\
D- & $\geq 60$ \\
F  & $<60$ \\
\end{tabular}
\end{table}

\subsection{Assignments}

There will be 4 assignments that will consist of both written 
exercises as well as \emph{substantial} programming problems.
You will be expected to follow all instructions on the 
assignments. Clarity and legibility are of great importance. 
If homework is sloppy or unclear, points may be deducted. You 
are not required to typeset your written solutions; however, 
it is strongly recommended that you do so using \LaTeX, 
markdown or similar typesetting system. Resources for \LaTeX\ 
are available on the course web page. Source code and all 
relevant files for programming portions must be handed in 
using the CSE web handin program.  Each assignment will 
have a fixed deadline based on CSE's server time.  No late
assignments will be accepted. 

Further, programming solutions will be graded using our online
webgrader system.  Failure to submit compilable or runnable 
code may result in a zero.  You are expected to do your own
substantial testing (and to submit valid, working test cases
as well), but it is essential that your submissions work on
the webgrader.

\subsection{Exams}

There will be two exams, both of which will be open-book, open-note,
open-computer but you may \emph{not} collaborate with anyone
in or outside the class on the solutions.

%\subsection{Quizzes}
%
%There will be 4 quizzes each of equal weight. They will generally 
%be short and will cover recent topics.

\subsection{Grading Policy}

If you have questions about grading or believe that points were 
deducted unfairly, you must first address the issue with the 
individual who graded it to see if it can be resolved.  Such 
questions should be made within a reasonable amount of time 
after the graded assignment has been returned.  No further 
consideration will be given to any assignment a week after 
it grades have been posted.  It is important to emphasize that 
the goal of grading is consistency.  A grade on any given 
assignment, even if it is low for the entire class, should 
not matter that much.  Rather, students who do comparable 
work should receive comparable grades (see the subsection 
on the scale used for this course).

\subsection{Late Work Policy}

In general, there will be no make-up exams or late work
accepted.  Exceptions may be made in certain circumstances 
such as health or emergency, but you must make every effort 
to get prior permission.  Documentation may also be required.

Homework assignments have a strict due date/time as defined by
the CSE server's system clock.  All program files must be handed
in using CSE's webhandin as specified in individual assignment
handouts.  Programs that are even a few seconds past the due 
date/time will be considered late and you will be locked out
of handing anything in after that time.  

\subsection{Webgrader Policy}

Failure to adhere to the requirements of an assignment in such 
a manner that makes it impossible to grade your program via 
the webgrader means that a disproportionate amount of time 
would be spent evaluating your assignment.  For this reason, 
we will not grade any assignment that does not compile and 
run through the webgrader.  

\subsection{Academic Integrity}

All homework assignments, programs, and exams must represent
your own work unless otherwise stated.  No collaboration with 
fellow students, past or current, is allowed unless otherwise 
permitted on specific assignments or problems.  The Department of
Computer Science \& Engineering has an Academic Integrity Policy.  
All students enrolled in any computer science course are bound 
by this policy.  You are expected to read, understand, and follow 
this policy.  Violations will be dealt with on a case by case 
basis and may result in a failing assignment or a failing grade 
for the course itself.  The most recent version of the Academic 
Integrity Policy can be found at \url{http://cse.unl.edu/academic-integrity}

\section{Summer Session Policy}

As a summer session, the course pace and presentation is 
accelerated.  What would normally be covered over 15 weeks 
is compressed into less than five.  Your success in this 
course depends on your acceptance of this fact and a 
commitment to putting in the extra work necessary to 
understand this material in the time that we do have.  
This means extensive daily review of materials outside 
of lecture and a diligent attitude toward completing 
assignments.  As such, no late work will be accepted and 
no makeup quizzes or exams will be given.  The compressed 
time period and logistics of offering summer courses make 
it extraordinarily difficult to make such considerations.

In addition, the summer version of this course lacks the 
same resources that would be available during the regular 
academic year.  In particular, the Student Resource Center 
is closed and there is no recitation section.  This may 
make getting additional help more difficult and you should 
make the appropriate adjustments or reconsider taking this 
course during the regular academic year. 

\section{Communication \& Getting Help}

The primary means of communication for this course is Piazza, an online
forum system designed for college courses.  We have established a Piazza 
group for this course and you should have received an invitation to join.
If you have not, contact the instructor immediately.  With Piazza you 
can ask questions anonymously, remain anonymous to your classmates, or 
choose to be identified.  Using this open forum system the entire class 
benefits from the instructor and TA responses.  In addition, you and 
other students can also answer each other's questions (again you may
choose to remain anonymous or identify yourself to the instructors or
everyone).  You may still email the instructor or TAs, but more than 
likely you will be redirected to Piazza for help.

In addition, there are two anonymous suggestion boxes that you may 
use to voice your concerns about any problems in the course if you 
do not wish to be identified.  My personal box is available at
\url{https://cse.unl.edu/~cbourke/email/}.  The department also 
maintains an anonymous suggestion box available at 
\url{https://cse.unl.edu/contact-form}.

\subsection{Getting Help}

Your success in this course is ultimately your responsibility.  Your
success in this course depends on how well you utilize the opportunities
and resources that we provide.  There are numerous outlets for learning
the material and getting help in this course:
\begin{itemize}
  \item Lectures: attend lectures regularly and when you do use the 
  time appropriately.  Do not distract yourself with social media or other
  time wasters.  Actively take notes (electronic or hand written).  It is
  well-documented that good note taking directly leads to understanding and
  retention of concepts.
  \item Required Reading: do the required reading on a regular basis.  The
  readings provide additional details and depth that you may not necessarily
  get directly in lecture.  
  \item Piazza: if you have questions ask them on Piazza.  It is the best and
  likely fastest way to get help with your questions.  Also, be sure to read
  other student's posts and questions and feel free to answer yourself!
  \item Office Hours: the instructor and GTA(s) hold regular office 
  hours throughout the week as posted on the
  course website.  Attend office hours if you have questions or want to 
  review material.  
  \item Don't procrastinate.  The biggest reason students fail this course
  is because they do not give themselves enough opportunities to learn the
  material.  Don't wait to the last minute to start your assignments.  Many
  people wait to the last minute and flood the TAs and SRC, making it difficult
  to get help as the due date approaches.  Don't underestimate how much time 
  your assignment(s) will take and don't wait to the week before hand to get 
  started.  Ideally, you should be working on the problems as we are covering 
  them.
  \item Get help in the \emph{right way}: when you go to the instructor or
  TA for help, you must demonstrate that you have put forth a good faith 
  effort toward understanding the material.  Asking questions that clearly 
  indicate you have failed to read the required material, have not been
  attending lecture, etc.\ is \emph{not acceptable}.  Don't ask generic
  questions like ``I'm lost, I don't know what I'm doing''.  Instead, 
  explain what you have tried so far.  Explain why you think what you 
  have tried doesn't seem to be working.  Then the TA will have an 
  easier time to help you identify misconceptions or problems.  This 
  is known as ``Rubber Duck Debugging'' where in if you try to explain 
  a problem to someone (or, lacking a live person, a rubber duck), 
  then you can usually identify the problem yourself.  Or, at the very 
  least, get some insight as to what might be wrong.
\end{itemize}



\end{document}
